\clearpage{\pagestyle{empty}\cleardoublepage}

\chapter*{Conclusioni}

\rhead[\fancyplain{}{\bfseries
CONCLUSIONI}]{\fancyplain{}{\bfseries\thepage}}
\lhead[\fancyplain{}{\bfseries\thepage}]{\fancyplain{}{\bfseries
CONCLUSIONI}}

\addcontentsline{toc}{chapter}{Conclusioni}

Questa tesi ha esplorato le potenzialità degli Open Data nella gestione e ottimizzazione dei servizi urbani, con particolare attenzione alla rete BolognaWiFi. L'obiettivo del progetto consisteva nello sviluppo di un'applicazione web interattiva per la visualizzazione dei dati relativi all'utilizzo della rete Wi-Fi pubblica del Comune di Bologna. Attraverso la creazione di un'interfaccia intuitiva e dinamica, è stato possibile rendere accessibili e interpretabili le informazioni raccolte, fornendo uno strumento utile per amministratori pubblici, ricercatori e cittadini.

Questa visualizzazione ha permesso di individuare a colpo d'occhio tendenze significative nell'utilizzo della rete, evidenziando pattern di affluenza, affollamento e spostamenti. La rappresentazione grafica di queste informazioni si è rivelata essenziale per facilitare la comprensione e l'elaborazione di strategie di miglioramento della connettività urbana. L'adozione di tecnologie moderne come Vue.js e Leaflet ha garantito un'interfaccia interattiva e altamente performante, migliorando l'esperienza utente e rendendo la navigazione più fluida ed efficace.

Il lavoro svolto ha dimostrato come l'uso intelligente degli Open Data possa migliorare significativamente la gestione e la pianificazione urbana. L'integrazione di strumenti di visualizzazione avanzati consente di sfruttare al meglio il potenziale dei dati pubblici, offrendo benefici concreti per la collettività. Grazie a questo progetto, è stato possibile non solo facilitare l'accesso alle informazioni sulla rete BolognaWiFi, ma anche proporre un modello replicabile per altre realtà urbane interessate alla valorizzazione dei propri dati.

Pur avendo raggiunto gli obiettivi prefissati, il progetto potrebbe essere ulteriormente ampliato e migliorato sotto diversi aspetti. Innanzitutto potremmo raffinare l'interfaccia utente, aggiungendo un grafico relativo all'andamento giornaliero, mensile o annuale, che per affollamento e affluenza sarebbe relativo a una singola zona, mentre per gli spostamenti, siano essi totali o mediani, andrebbe considerato un singolo movimento direzionato da una certa area verso un'altra. Riguardo agli spostamenti, si potrebbe calcolare il flusso in entrata e in uscita per ciascuna zona, assegnando dei relativi colori in scala cromatica per ogni area invece dell'attuale tinta monocromatica.

Inoltre, è importante notare come l'applicazione da noi creata permetta già adesso una possibile estensione ad altre città, aggiungendo i relativi dati sul database e modificando le coordinate iniziali della visualizzazione su mappa. L'unica vera esigenza sarebbe quella di adattare le API attuali per raccogliere i dati da una nuova banca dati di Open Data relativi alla città in cui si vuole estendere il servizio.

L'adozione di queste migliorie potrebbe trasformare l'applicazione sviluppata in un riferimento per la gestione e l'analisi dei dati urbani, promuovendo un uso più consapevole ed efficace delle risorse digitali a disposizione delle città. In questo modo potrebbe costituire un punto di partenza per ulteriori sviluppi nell'ambito della gestione delle infrastrutture digitali delle Smart Cities, contribuendo a rendere le città più connesse, intelligenti e a misura di cittadino.