\clearpage{\pagestyle{empty}\cleardoublepage}

\chapter*{Conclusioni}

\rhead[\fancyplain{}{\bfseries
CONCLUSIONI}]{\fancyplain{}{\bfseries\thepage}}
\lhead[\fancyplain{}{\bfseries\thepage}]{\fancyplain{}{\bfseries
CONCLUSIONI}}

\addcontentsline{toc}{chapter}{Conclusioni}

Questa tesi ha esplorato le potenzialità degli Open Data nella gestione e ottimizzazione dei servizi urbani, con particolare attenzione alla rete BolognaWiFi. L'obiettivo del progetto era sviluppare un'applicazione web interattiva per la visualizzazione e l'analisi dei dati relativi all'uso della rete WiFi pubblica del Comune di Bologna. Attraverso lo sviluppo di un'interfaccia intuitiva e dinamica, è stato possibile rendere accessibili e interpretabili le informazioni raccolte, fornendo uno strumento utile per amministratori pubblici, ricercatori e cittadini.

L'analisi dei dati raccolti ha permesso di individuare tendenze significative nell'utilizzo della rete, evidenziando pattern di affluenza e spostamento. La rappresentazione grafica di queste informazioni si è rivelata essenziale per facilitare la comprensione e l'elaborazione di strategie di miglioramento della connettività urbana. L'adozione di tecnologie moderne come Vue.js e Leaflet ha garantito un'interfaccia interattiva e altamente performante, migliorando l'esperienza utente e rendendo la navigazione più fluida ed efficace.

Uno degli aspetti più rilevanti emersi durante il progetto è stata la necessità di un'integrazione sempre più spinta tra i dati della rete WiFi e altri dataset relativi alla mobilità urbana, al turismo e alla gestione dei servizi pubblici. Questa sinergia potrebbe rappresentare un passo avanti nella costruzione di strumenti predittivi basati sull'analisi dei dati in tempo reale, favorendo una pianificazione urbana più intelligente e reattiva alle esigenze dei cittadini.

\section*{Sviluppi futuri}

Pur avendo raggiunto gli obiettivi prefissati, il progetto può essere ulteriormente ampliato e migliorato sotto diversi aspetti. Tra le possibili evoluzioni future si possono individuare le seguenti direzioni:

\begin{itemize}
    \item \textbf{Integrazione con ulteriori dataset}: Espandere l'applicazione integrando dati relativi alla mobilità cittadina, all'occupazione degli spazi pubblici e ai flussi turistici, per fornire una visione più completa e dettagliata del comportamento urbano.
    \item \textbf{Miglioramento delle prestazioni}: Ottimizzare il caricamento dei dati e la reattività dell'interfaccia per garantire una maggiore scalabilità dell'applicazione anche in presenza di volumi elevati di informazioni.
    \item \textbf{Implementazione di modelli predittivi}: Sviluppare algoritmi di machine learning per analizzare i dati storici e prevedere i futuri trend di utilizzo della rete WiFi, supportando così le decisioni strategiche delle amministrazioni pubbliche.
    \item \textbf{Espansione del progetto ad altre città}: Applicare il modello sviluppato in questa tesi ad altre città, adattandolo alle specifiche esigenze territoriali e alle diverse infrastrutture di rete disponibili.
    \item \textbf{Miglioramento dell'interfaccia utente}: Raffinare l'esperienza utente attraverso test di usabilità e raccolta di feedback per rendere l'applicazione ancora più intuitiva e accessibile.
\end{itemize}

L'adozione di queste migliorie potrebbe trasformare l'applicazione sviluppata in un riferimento per la gestione e l'analisi dei dati urbani, promuovendo un uso più consapevole ed efficace delle risorse digitali a disposizione delle città.

\section*{Conclusione generale}

In conclusione, il lavoro svolto ha dimostrato come l'uso intelligente degli Open Data possa migliorare significativamente la gestione e la pianificazione urbana. L'integrazione di strumenti di visualizzazione avanzati consente di sfruttare al meglio il potenziale dei dati pubblici, offrendo benefici concreti per la collettività. Grazie a questo progetto, è stato possibile non solo facilitare l'accesso alle informazioni sulla rete BolognaWiFi, ma anche proporre un modello replicabile per altre realtà urbane interessate alla valorizzazione dei propri dati.

L'auspicio è che questa ricerca possa costituire un punto di partenza per ulteriori sviluppi nell'ambito della gestione delle infrastrutture digitali cittadine, contribuendo a rendere le città più connesse, intelligenti e a misura di cittadino.
