\pagenumbering{roman}
\chapter*{Introduzione}

\rhead[\fancyplain{}{\bfseries
INTRODUZIONE}]{\fancyplain{}{\bfseries\thepage}}
\lhead[\fancyplain{}{\bfseries\thepage}]{\fancyplain{}{\bfseries
INTRODUZIONE}}

\addcontentsline{toc}{chapter}{Introduzione}

Negli ultimi anni, il concetto di Open Data ha acquisito una sempre maggiore rilevanza, trasformandosi in un elemento chiave per l'innovazione, la trasparenza amministrativa e il miglioramento dei servizi pubblici. Le amministrazioni locali e nazionali, il settore privato e la ricerca accademica stanno investendo sempre più nell'analisi e nella gestione di questi dati per ottimizzare la pianificazione urbana e offrire ai cittadini servizi più efficienti e accessibili. In questo scenario, il progetto BolognaWiFi rappresenta un'iniziativa significativa, volta a fornire connettività pubblica gratuita e a raccogliere dati sull'uso della rete per comprendere le dinamiche della città e migliorare l'infrastruttura digitale.

L'importanza degli Open Data risiede nella loro capacità di trasformare il processo decisionale basandolo su dati concreti e aggiornati. Nel caso della rete BolognaWiFi, i dati raccolti permettono di analizzare l'affluenza in specifiche aree urbane, monitorare i flussi di connessione e identificare le esigenze degli utenti. Tuttavia, la grande quantità di dati raccolti presenta la sfida della loro interpretazione. Per questo motivo, è essenziale sviluppare strumenti di visualizzazione che consentano di trasformare i dati grezzi in informazioni facilmente comprensibili e fruibili.

\section*{Obiettivo della tesi}

Il presente lavoro di tesi si pone come obiettivo principale la progettazione e lo sviluppo di un'applicazione web interattiva per la visualizzazione dei dati relativi all'utilizzo della rete BolognaWiFi. L'intento è quello di rendere accessibili e interpretabili le informazioni raccolte, fornendo un'interfaccia intuitiva per utenti finali e amministratori pubblici. In particolare, l'applicazione permette di:

\begin{itemize}
    \item Visualizzare graficamente l'andamento dell'affluenza e dell'utilizzo della rete WiFi nelle diverse zone della città;
    \item Analizzare i dati in modo interattivo attraverso filtri temporali e geografici;
    \item Integrare tecnologie di visualizzazione avanzate, come Vue.js e Leaflet, per garantire un'esperienza utente fluida e accessibile;
    \item Supportare le decisioni strategiche per l'ottimizzazione dei servizi pubblici attraverso l'analisi dei dati di connessione.
\end{itemize}

\section*{Risultati raggiunti}

La realizzazione dell'applicazione ha consentito di dimostrare l'efficacia della visualizzazione dati nell'ambito della gestione urbana. Grazie all'implementazione di strumenti interattivi, l'applicazione sviluppata consente di identificare trend significativi e di migliorare la comprensione dei dati da parte degli utenti non esperti. Tra i principali risultati raggiunti si evidenziano:

\begin{itemize}
    \item La creazione di un'interfaccia dinamica e intuitiva che permette agli utenti di esplorare i dati relativi alla rete WiFi;
    \item L'ottimizzazione delle prestazioni del sistema attraverso l'utilizzo di tecnologie scalabili e performanti;
    \item L'integrazione di tecniche di analisi che facilitano la comprensione dei pattern di utilizzo e affluenza;
    \item Il supporto alla gestione delle risorse pubbliche grazie alla possibilità di monitorare in tempo reale il funzionamento della rete WiFi comunale.
\end{itemize}

Questi risultati dimostrano l'importanza di strumenti interattivi di visualizzazione dei dati per il miglioramento della gestione urbana, fornendo un modello replicabile per altre amministrazioni e contesti.

\section*{Struttura della tesi}

Il seguito della tesi è così organizzato:

\begin{itemize}
    \item \textbf{Capitolo 1}: Introduce il concetto di Open Data e il loro ruolo nella gestione urbana, con un approfondimento sul progetto BolognaWiFi e il contesto in cui si inserisce.
    \item \textbf{Capitolo 2}: Descrive le tecnologie utilizzate per lo sviluppo dell'applicazione, tra cui HTML, CSS, JavaScript, Vue.js, Leaflet e database SQL, evidenziandone le caratteristiche e i vantaggi.
    \item \textbf{Capitolo 3}: Presenta il processo di progettazione e implementazione della web app, con particolare attenzione alla struttura del database, all'architettura software e alla connessione tra frontend e backend.
    % \item \textbf{Conclusioni}: Sintetizza i risultati ottenuti e propone possibili sviluppi futuri, con particolare attenzione alle potenzialità dell'integrazione con altri dataset e alle prospettive di miglioramento dell'applicazione.
\end{itemize}

Questa organizzazione consente di offrire una panoramica completa del lavoro svolto, evidenziando il contributo originale del progetto e le sue implicazioni future nel contesto dell'uso degli Open Data per la gestione urbana.

% \newpage
\clearpage{\pagestyle{empty}\cleardoublepage}