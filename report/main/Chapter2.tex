\clearpage{\pagestyle{empty}\cleardoublepage}
\chapter{Tecnologie utilizzate}
In questo capitolo presenteremo le varie tecnologie utilizzate per la realizzazione del progetto, descrivendole in modo da fornire una conoscenza di base abbastanza solida da permettere la comprensione dell'ambiente.

\section{HTML}
HTML, acronimo che significa \textit{HyperText Markup Language}, è un linguaggio di markup che permette di impaginare e formattare pagine web collegate tra loro tramite link. Un ipertesto non è altro che l'albero di navigazione che collega le pagine web, ovvero un flusso infinito di pagine collegate tra loro attraverso dei link che permettono di spostarsi da un contenuto all'altro \cite{HTML, HTML_Mozilla}.

HTML risponde all'esigenza di riuscire a pubblicare del testo online, mantenendo la formattazione e il significato di ciascuna delle sue parti, utilizzando dei marcatori detti \Verb_<tag>_. Il browser legge il tag e il suo contenuto, quindi traduce a schermo il codice usando i criteri specificati da HTML. Questo ha permesso di costruire pagine con  una struttura simile fra di loro e soprattutto replicabile seguendo uno standard. Nel giro di pochi anni furono aggiunti via via sempre più tag ed elementi per consentire la creazione di pagine contenenti immagini, elementi interattivi, form, pulsanti, tabelle e altri ancora. Pertanto, HTML si è trasformato in un linguaggio molto completo ma in continuo mutamento, che oggi viene mantenuto dal World Wide Web Consortium (W3C), una associazione non governativa che si occupa di implementare nuove funzioni per rendere il web sempre più libero e accessibile.

Riguardo ai tag, sono alla base dell'HTML e ciascuno di essi corrisponde a un determinato tipo di contenuto. Ogni tag può avere degli attributi specifici, cosa che permette di costruire pagine diverse tra di loro e in modo tale che rispondano alle necessità di chi le scrive. Le pagine in HTML hanno una struttura ad albero: proseguendo lungo la ramificazione, si possono trovare più o meno elementi che costruiscono la pagina stessa seguendo una precisa gerarchia. Ad esempio, nel seguente frammento di codice:
\begin{verbatim}
    <html>
        <head>
            <title>Titolo</title>
        </head>
        <body>
            <p>Paragrafo</p>
        </body>
    </html>
\end{verbatim}
\Verb_<html>_ indica l'inizio della parte di codice che verrà espressa utilizzando il linguaggio HTML. Tranne alcuni tag detti \textit{self-closing tags}, tutti vanno chiusi mediante il rispettivo tag di chiusura, che in questo caso è \Verb_</html>_.
\Verb_<head>_ specifica l'header della pagina, il quale racchiude delle informazioni importanti per il suo funzionamento, ma invisibili dal nostro dispositivo. Contiene a sua volta un \Verb_<title>_, ovvero un titolo che è quello rappresentativo della pagina stessa e del suo contenuto. Una volta chiusa l'intestazione, si passa al \Verb_<body>_, cioè il contenuto della pagina. Al suo interno troviamo il tag \Verb_<p>_, utilizzato per scrivere paragrafi di testo, al cui interno viene scritto il rispettivo contenuto. Tutti i vari browser web sono pensati per interpretare i tag più o meno ugualmente ogni volta, ma ci possono essere eccezioni in quanto ogni browser implementa un proprio rendering della pagina web.

HTMl non è un linguaggio di programmazione, bensì un linguaggio di markup: descrive al browser com'è fatta la struttura di una pagina, e niente più. Un linguaggio di programmazine, invece, ha un ruolo funzionale: risolve cicli di codice seguendo una struttura fatta di \Verb_if_ e \Verb_else_, può svolgere calcoli matematici, può manipolare dati e variabili. Quindi, è il browser web che è programmato per capire la struttura delle pagine scritte in HTML, mentre quest'ultimo descrive soltanto la struttura della pagina e del suo contenuto. Di conseguenza, HTML non è un linguaggio di programmazione, mentre lo sono ad esempio JavaScript e PHP.

Ad oggi non è più sufficiente utilizzare solo HTML per realizzare contenuti web, in quanto le esigenze sono da tempo cambiate: ai siti web serve più di soli contenuti testuali, quindi vengono usati anche altri linguaggi, tra cui CSS e JavaScript, che consentono di scrivere pagine ricche e complete, sia nell'aspetto di grafica che nel comportamento \cite{HTML}.

\section{CSS}
CSS, acronimo di \textit{Cascading Style Sheets}, è un linguaggio di stile per i documenti web. I CSS forniscono istruzioni a un browser o a un altro programma utente su come il documento debba essere presentato all'utente, definendone proprietà come il font, i colori, le immagini di sfondo, il layout, il posizionamento delle colonne o di altri elementi della pagina \cite{CSS_Introduzione, CSS_Mozilla}.

CSS è un linguaggio nato per essere il complemento ideale di HTML, per questo i due linguaggi hanno sempre proceduto parallelamente: nelle intenzioni del W3C, HTML è un semplice linguaggio strutturale, estraneo a qualunque scopo che riguardi la presentazione di un documento. Invece, CSS è lo strumento designato proprio per arricchire l'aspetto visuale e la presentazione di una pagina, nato per separare il contenuto dalla presentazione \cite{CSS_Introduzione}.

Anche CSS è un \textit{living standard} come HTML e per questo viene regolarmente sviluppato dal W3C. Il suo compito principale è definire il design di un sito web e per questo scopo vengono assegnati determinati valori agli elementi HTML con l'aiuto delle proprietà del linguaggio di stile. Queste regole hanno una struttura di base che corrisponde allo schema \Verb_selettore { dichiarazione }_. Un selettore non è altro che una rappresentazione dell'elemento HTML al quale fa riferimento la regola, mentre la dichiarazione consiste dalla combinazione tra proprietà e valore che viene annotata tra due parentesi graffe. Ogni dichiarazione finisce con un punto e virgola, come ad esempio \Verb_h2 { color: #ff0000; }_, nel quale il selettore \Verb_h2_ rappresenta le intestazioni di secondo ordine, ovvero i sottotitoli. La dichiarazione, invece, utilizza la proprietà \Verb_color_ per colorare tali sottotitoli di rosso, in questo caso utilizzando un colore espresso in esadecimale per definire impostazioni di colore precise.

CSS supporta diversi selettori in grado di dare istruzioni di regole:
\begin{itemize}
    \item \Verb_selettore_\\
    Corrisponde al nome dell'elemento HTML a cui fa riferimento: lo stile viene applicato su tutti gli elementi HTML dello stesso tipo.
    \item \Verb_.selettore_\\
    Si rivolge a tutti gli elementi di una classe specifica e si scrive con un punto davanti al nome della classe HTML corrispondente.
    \item \Verb_#selettore_\\
    Si riferisce a un unico elemento con un ID specifico. L'integrazione nel codice sorgente HTML avviene grazie all'attributo id, ovvero \Verb_id="identificatore"_.
    \item \Verb_*_ \textit{(selettore universale)}\\
    Si rivolge a tutti gli elementi HTML di un documento.
\end{itemize}

Il CSS può essere dichiarato in vari modi nell'HTML:
\begin{itemize}
    \item \textbf{Inline}\\
    Viene definito direttamente sul tag dell'elemento HTML utilizzando l'attributo \Verb_style_. Ha il vantaggio che non è necessario configurare un foglio di stile apposito e ha la priorità elevata, ma nel momento in cui deve essere applicato un blocco di dichiarazioni questo metodo di stile diventa poco chiaro e ridondante.
    \item \textbf{Interno}\\
    Viene definito all'interno del tag \Verb_<style>_ nella \Verb_<head>_ del documento HTML.
    \item \textbf{Esterno}\\
    Viene definito in un foglio di stile separato, il quale verrà collegato al documento HTML di base utilizzando l'elemento HTML \Verb_<link>_, sempre nella \Verb_<head>_. Può essere riutilizzato in più pagine HTML, semplicemente importandolo ove necessario.
\end{itemize}

\section{SCSS}
SCSS, acronimo di \textit{Sassy Cascading Style Sheets}, è un'estensione di CSS molto popolare che aggiunge funzionalità potenti che aumentano le capacità standard di CSS \cite{SASS, SCSS_SASS}.

In qualità di linguaggio di script preprocessing, SCSS permette agli sviluppatori di utilizzare variabili, regole annidate (\textit{nested}), \textit{mixins} e funzioni, che semplificano il processo di scrittura e mantenimento di stylesheets complessi. Essendo un superset di CSS, SCSS è pienamente compatibile con CSS e rende più facile creare codice pulito, organizzato e riutilizzabile, migliorando l'efficienza e la scalabilità dei progetti di web development.


SCSS è una delle due possibili sintassi per il preprocessore SASS, o \textit{Syntactically Awesome Style Sheets}. Come ogni preprocessore, SASS viene compilato in codice CSS nativo che funziona su ogni web browser. La vera potenzialità di questo linguaggio si vede sul lato sviluppatore, in quanto permette di scrivere codice SCSS conciso che verrà compilato in un codice CSS più lungo. Per questo, gli sviluppatori possono riuscire a fare di più scrivendo meno, senza compromettere la compatibilità col web.

La differenza principale tra SCSS e CSS è che SCSS è una sintassi per il preprocessore SASS, mentre CSS è un linguaggio di stile che descrive come un browser debba visualizzare gli elementi HTML. SCSS supporta da molto tempo l'utilizzo di variabili in qualsiasi punto dello stylesheet, mentre CSS offre un supporto nativo per le variabili che è relativamente recente e può essere usato solo per salvare valori e tokens della UI. In SCSS, la sintassi può essere annidata, permettendo di effettuare il nesting sia delle proprietà che di altri selettori, mentre in CSS un singolo selettore può annidare proprietà ma non altri selettori. Inoltre, SCSS supporta i mixins, che sono gruppi di dichiarazioni CSS che possono essere riutilizzate all'interno dello stylesheet, il che può essere utile quando utilizziamo prefissi vendor (come -webkit- per Chrome o -moz- per Firefox), animazioni complesse, o altro codice che potrebbe essere riutilizzato in più posti.

La sintassi SCSS e CSS hanno delle somiglianze in comune, ma quella SCSS permette l'utilizzo di funzionalità più avanzate. SCSS utilizza i punti e virgola e l'indentazione per mantenere la formattazione, mentre il normale CSS potrebbe essere più flessibile senza tuttavia offrire le funzionalità avanzate di SCSS. L'utilizzo di SCSS permette di mantenere una codebase più pulita e più facilmente mantenibile.

Per compilare SCSS in CSS si utilizza un compiler SASS, che funziona con entrambe le sintassi SASS e SCSS \cite{SCSS}. Queste due sintassi sono equivalenti, l'unica differenza è che in SASS si elimina l'utilizzo di punti e virgola e parentesi graffe mediante una sintassi \textit{whitespace-sensitive}, rendendolo però incompatibile con CSS \cite{SASS}.

\section{JavaScript}
JavaScript è un linguaggio di scripting, o di programmazione, che permette di implementare funzionalità complesse all'interno delle pagine web, ad esempio ogni volta che un sito debba essere dinamico e non statico, per creare contenuti aggiornati automaticamente, mappe interattive, controllare elementi multimediali, oppure grafiche animate in 2D o 3D. Costituisce il terzo strato dello stack delle tecnologie web standard \cite{JavaScript, JavaScript_Mozilla}.

Lato client, il linguaggio JavaScript consiste delle solite funzionalità di programmazione come il salvataggio di valori in variabili, le operazioni su stringhe, e l'esecuzione di codice in risposta a certi eventi che avvengono su una pagina web, come quello di \Verb_click_. Ma ancora più importante, ci permette di utilizzare le funzionalità costruite su altro codice JavaScript client-side, mediante le \textit{Application Programming Interfaces}, o APIs. Queste sono blocchi di codice già pronto che consentono a uno sviluppatore di implementare progammi che altrimenti sarebbero difficili o impossibili da implementare, rendendo questi compiti più semplici da realizzare. Le API possono essere built-in nel browser o di terze parti. Queste ultime verranno approfondite in una sezione successiva. Le API del browser permettono di esporre dati dall'ambiente del computer, o effettuare compiti complessi. Per esempio:
\begin{itemize}
    \item \textbf{DOM (\textit{Document Object Model}) API}\\
    Permettono di manipolare HTML e CSS, creando, rimuovendo e cambiando l'HTML, applicando nuovi stili dinamicamente alla pagina, ecc. Per esempio, vengono utilizzate ogni volta che vediamo aprirsi una finestra popup su una pagina, o quando viene visualizzato del nuovo contenuto.
    \item \textbf{Geolocation API}\\
    Permettono di recuperare informazioni geografiche, come la posizione GPS corrente.
    \item \textbf{Canvas e WebGL API}\\
    Permettono di creare animazioni 2D e 3D.
    \item \textbf{Audio e Video API}\\
    Permettono di riprodurre audio e video direttamente in una pagina web, o filmare un video dalla webcam e visualizzarlo su un altro computer.
\end{itemize}

JavaScript è un linguaggio leggermente interpretato: il browser riceve il codice JS originale e lancia lo script, e nei browser moderni viene utilizzata una tecnica di compilazione detta \textit{just-in-time} per migliorare le performance. In questo caso, il codice JS viene compilato in un formato binario più veloce mentre lo script è in esecuzione, così da lanciarlo il più rapidamente possibile. Tuttavia, JavaScript è comunque considerato un linguaggio interpretato, dato che la compilazione avviene a \textit{run-time}, invece che \textit{ahead-of-time}.

JavaScript può essere utilizzato sia client-side che server-side. Lato client, viene eseguito sul computer dell'utente, ad esempio quando visualizziamo la pagina web, viene scaricato il codice client-side del sito web, quindi è eseguito e visualizzato dal browser. Lato server, invece, viene eseguito dal server, per poi inviare i risultati al browser lato client, dove verranno scaricati e visualizzati. Tra i vari linguaggi server, possiamo utilizzare JavaScript nell'ambiente Node.js.

Inoltre, è utilizzato per creare pagine web dinamiche, dove \textit{dinamico} si riferisce all'abilità, sia in client- che server-side JS, di aggiornare la visualizzazione di una webpage o webapp per mostrare cose diverse in circostanze diverse, generando il nuovo contenuto richiesto. Il lato server genera dinamicamente il nuovo contenuto, prendendo i dati dal database, mentre lato client JavaScript crea dinamicamente nuovo contenuto all'interno del browser, prendendo i dati che gli sono stati inviati dal backend. Generalmente questi due approcci lavorano insieme. Una pagina web statica, invece, non aggiorna il contenuto dinamicamente e mostra sempre la stessa schermata \cite{JavaScript}.

\subsection{AJAX}
AJAX, acronimo di \textit{Asynchronous JavaScript and XML}, è una tecnica di sviluppo web in cui una webapp scarica contenuto dal server mediante una richiesta HTTP asincrona, quindi lo utilizza per aggiornare le parti della pagina rilevanti, senza bisogno di ricaricare nuovamente la pagina. Questo può rendere la pagina più responsive, in quanto si richiedono solo le parti che necessitano di essere aggiornate.

AJAX può essere utilizzato per creare \textit{single-page apps}, dove l'intera webapp consiste in un singolo documento che usa AJAX per aggiornare il contenuto in base alle richieste.

Inizialmente, AJAX venne implementato utilizzando l'interfaccia \\\Verb_XMLHttpRequest_, ma l'API di \Verb_fetch()_ è più adatta alle applicazioni web moderne, in quanto è più potente, flessibile, e in grado di integrarsi meglio con le tecnologie web fondamentali come i \textit{service workers}. I framework web moderni offrono astrazioni per AJAX. Tuttavia, ad oggi questa tecnica è talmente diffusa che il termine specifico \textit{AJAX} viene utilizzato raramente \cite{AJAX}.

\subsection{Framework e librerie}
In JavaScript sono disponibili un gran numero di framework e librerie, sviluppati nel corso degli anni per velocizzare e semplificare lo sviluppo web.

Le librerie sono dei pacchetti di classi e funzioni riutilizzabili per risolvere particolari problemi, sono flessibili e possono essere utilizzate liberamente. Alcuni esempi di librerie famose sono:
\begin{itemize}
    \item \textbf{React.js}, utilizzato per creare UI complesse e gestire lo stato.
    \item \textbf{jQuery}, utilizzato per gestire eventi JS ed effettuare chiamate AJAX.
    \item \textbf{D3.js} (\textit{Data-Driven Documents}), utilizzato per la visualizzazione dati.
    \item \textbf{Chart.js}, utilizzato per disegnare vari tipi di grafici utilizzando l'elemento HTML5 \Verb_<canvas>_.
\end{itemize}

I framework, invece, offrono un approccio strutturato allo sviluppo di un progetto, mediante un insieme di strumenti utili a creare applicazioni in maniera standardizzata. Specificano dove inserire quali frammenti di codice e spesso includono funzionalità come la gestione del routing, richieste HTTP, o dei dati di backend. Questo approccio strutturato semplifica lo sviluppo e assicura consistenza tra i vari progetti. Alcuni esempi di framework famosi sono:
\begin{itemize}
    \item \textbf{Angular}, sviluppato da Google, utilizzato per lo sviluppo cross-platform.
    \item \textbf{Vue.js}, utilizzato per creare UI e SPAs (\textit{Single-Page Applications}).
    \item \textbf{Next.js}, framework basato su React per sviluppare webapp multipiattaforma \cite{JS_Frameworks_Libraries}.
\end{itemize}

\`E possibile anche eseguire codice JavaScript al di fuori dal browser, utilizzando l'ambiente di esecuzione Node.js, che è open-source e multipiattaforma \cite{Node.js}.

\section{Leaflet}
Leaflet è una libreria JavaScript open-source che permette di visualizzare mappe interattive mobile-friendly. Il suo codice pesa solo 42KB, ma contiene tutte le funzionalià di mapping che possono servire a quasi tutti gli sviluppatori \cite{Leaflet_Overview, Leaflet}.

Questa libreria è stata progettata avendo in mente la semplicità, le prestazioni e l'usabilità. Funziona in modo efficiente su tutte le principali piattaforme desktop e mobile, permette l'estensione delle sue funzionalità mediante l'utilizzo di plugins e offre diverse API.

Leaflet non cerca di fare tutto, ma si focalizza sulle funzionalità di base per fare in modo che funzionino perfettamente. Le sue caratteristiche principali sono:
\begin{itemize}
    \item \textbf{Layers}\\
    Offre i cosiddetti \textit{tile layers}, i \textit{markers}, e i popup. Inoltre, si possono mettere delle immagini come overlays e i GeoJSON. Sulla mappa si possono disegnare poligoni, polylines e cerchi.
    \item \textbf{Interazione}\\
    Supporta lo scorrimento, lo zoom tramite gesture, doppio click o scroll wheel, lo spostamento dei marker, gli eventi, e altro ancora.
    \item \textbf{Funzionalità}\\
    Visivamente, supporta le animazioni per lo zoom e il fade. I popup possono essere customizzati tramite CSS, e si può creare un marker personalizzato tramite immagini o HTML. Riguardo alle mappe, gli sviluppatori possono utilizzare i layer che preferiscono.
    \item \textbf{Controlli}\\
    Sono presenti bottoni per lo zoom e il cambio di layer, un'indicatore per la scala, e uno per l'attribuzione \cite{Leaflet_Overview}.
\end{itemize}

\section{Vue.js}
Vue è un framework JavaScript utilizzato per la costruzione di interfacce utente (UI). Si basa sullo standard stack HTML, CSS e JavaScript e offre un modello di programmazione dichiarativo e basato sui componenti, che aiuta a sviluppare in modo efficiente delle UI anche complesse \cite{Vue}. Le funzionalità chiave di Vue sono due:
\begin{itemize}
    \item \textbf{Rendering dichiarativo}\\
    Vue estende lo standard HTML mediante una sintassi di template che permettono di descrivere output HTML in modo dichiarativo, basato su uno stato definito in JavaScript.
    \item \textbf{Reattività}\\
    Vue traccia automaticamente i cambiamenti di stato in JavaScript e aggiorna il DOM in modo efficiente quando avvengono queste variazioni.
\end{itemize}

Vue è un framework ed ecosistema che copre la maggioranza delle più comuni funzionalità richieste nello sviluppo frontend. Tuttavia, il web è estremamente vario e ciò che viene costruito può cambiare drasticamente in forma e scala. Vue è stato disegnato proprio per rispondere a queste esigenze, in modo da essere flessibile e adottabile in modo incrementale. In base al caso d'uso, Vue può essere utilizzato in diverse modalità:
\begin{itemize}
    \item Miglioramento di HTML statico senza necessità di compilazione.
    \item Integrazione di componenti web su qualsiasi pagina.
    \item Creazione di \textit{Single-Page Applications} (SPA).
    \item Utilizzo di Fullstack o \textit{Server-Side Rendering} (SSR).
    \item Utilizzo di Jamstack o \textit{Static Site Generation} (SSG).
    \item Targeting specifico di desktop, mobile, WebGL, o anche il terminale.
\end{itemize}
Vue è flessibile, ma le sue nozioni principali sono condivise a tutti i livelli, rendendolo un framework adattabile alle esigenze di chiunque \cite{Vue}.

% TODO: add Vite

\section{NPM}
NPM, o \textit{Node Package Manager}, è il package manager di default per Node.js. Nel settembre 2022, il registro di NPM conteneva oltre 2.1 milioni di pacchetti, rendendolo il più grande repository al mondo di codice che utilizza un solo linguaggio, possiamo quindi essere quasi sicuri che esista un pacchetto per ogni evenienza. NPM nacque come un modo per scaricare e gestire le dipendenze di Node.js, ma è in seguito diventato uno strumento utilizzato anche nel JavaScript a frontend \cite{NPM_Package_Manager, NPM_Docs}.

NPM installa, aggiorna e gestisce i download delle dipendenze del progetto. Le dipendenze sono blocchi di codice precostruiti, come le librerie e i pacchetti, di cui l'applicazione ha bisogno per poter funzionare.

Oltre ai normali download, NPM gestisce anche il versioning, così da poter specificare qualsiasi versione specifica di un pacchetto, o richiedere una versione più recente o più vecchia di quella che ci serve. Spesso capita che una libreria sia compatibile solamente con una major release di un'altra libreria, o che un bug nell'ultima versione di una libreria sia ancora irrisolto e continui a causare problemi. Specificare esplicitamente una versione di una libreria aiuta anche a tenere tutto il team sulla stessa versione di un package, così che tutti la utilizzino fino a quando il file \Verb_package.json_ sia aggiornato. In tutti questi casi, il controllo delle versioni aiuta notevolmente e NPM segue lo stesso standard di \textit{semantic versioning}, o \textit{semver}.

NPM consente anche agli sviluppatori di eseguire comandi personalizzati, supportando un formato per specificare task a terminale che vengono lanciati come scripts. Per esempio, utilizzando \Verb_npm run dev_ possiamo lanciare l'app in modalità debug senza scrivere l'effettivo comando corrispondente, che sarebbe molto più lungo. Altri comandi, tra tutti, sono \Verb_npm run prod_ e \Verb_npm run build_ \cite{NPM_Package_Manager}.

\section{PHP}
PHP, acronimo ricorsivo per \textit{PHP: Hypertext Preprocessor}, è un linguaggio di scripting open-source e general-purpose molto usato, che è specialmente adatto per lo sviluppo web e ad essere integrato nell'HTML \cite{PHP, PHP_Docs}.

Invece di utilizzare molti comandi per visualizzare l'HTML, come avviene in C o Perl, le pagine PHP contengono HTML all'interno di codice integrato che svolge qualche preciso compito. Il codice PHP è racchiuso in speciali istruzioni di inizio e fine, ovvero \Verb_<?php_ e \Verb_?>_ che permettono di entrare e uscire dalla \textit{modalità PHP}.

Ciò che distingue PHP dal codice JavaScript client-side è che il codice è eseguito sul server, generando HTML che viene quindi inviato al client. Il client riceve così il risultato dell'esecuzione di quello script, senza sapere quale sia il codice sottostante. Un web server può essere configurato per processare tutti i file HTML utilizzando PHP, senza che gli utenti abbiano alcun modo per capire che si stia utilizzando PHP.

PHP riguarda principalmente lo scripting lato server, quindi può fare qualsiasi cosa che altri programmi CGI (\textit{Common Gateway Interface}) sanno fare, come raccogliere dati da un \textit{form}, generare dinamicamente il contenuto della pagina, o inviare e ricevere cookies. Due sono le aree principali dove gli script PHP vengono utilizzati:
\begin{itemize}
    \item \textbf{Scripting lato server}\\
    Questo è il più utilizzato e costituisce il principale campo di applicazione per PHP. Per funzionare, richiede un parser PHP (CGI o modulo server), un server web, e un browser web. Tutti questi possono essere eseguiti localmente per scopi di sviluppo.
    \item \textbf{Scripting a command line}\\
    Uno script PHP può essere eseguito senza richiedere alcun server o browser, basta solo disporre di un parser PHP per poterlo utilizzare in questa maniera. Questo tipo di utilizzo è ideale per script eseguiti regolarmente tramite \Verb_cron_, su Unix e macOS, o con \textit{Task Scheduler} su Windows. Questi script possono essere semplicemente usati anche per processare del testo.
\end{itemize}

PHP può essere utilizzato su tutti i principali sistemi operativi, tra cui Linux, molte varianti Unix come Solaris e OpenBSD, Microsoft Windows, macOS, RISC OS, e altri ancora. PHP supporta anche la maggior parte degli attuali web server, inclusi Apache, IIS, e molti altri, incluso qualsiasi web server in grado di utilizzare \textit{binaries} di FastCGI, come lighttpd e nginx. PHP è in grado di lavorare come modulo o come processore CGI.

Con PHP, gli sviluppatori hanno la libertà di scelta sia per il sistema operativo che per il web server. Inoltre, viene data la possibilità di scegliere se utilizzare un paradigma di programmazione procedurale o uno ad oggetti (OOP), o addirittura di mescolarli.

PHP non si limita a sfornare HTML, ma è in grado di generare file RTF (\textit{Rich File Types}), come immagini o file PDF, criptare dati, e inviare emails. Inoltre, può produrre facilmente qualsiasi tipologia di testo, come JSON o XML. PHP è in grado di generare questi file automaticamente e di salvarli a file system, invece di mostrarli a schermo, formando così una cache a lato server per il contenuto dinamico.

Una delle funzionalità più importanti di PHP è il suo supporto a una grande varietà di database. Scrivere una pagina web che acceda a un db diventa così incredibilmente semplice, utilizzando una delle estensioni specifiche come mysqli per MySQL, o mediante un livello di astrazione come PDO (\textit{Protected Destination of Origin}), oppore connettendosi a un database che supporta lo standard Open Database Connection tramite l'estensioen ODBC. Altri database, invece, possono utilizzare cURL o sockets.

PHP supporta molti protocolli per la comunicazione con altri servizi, ad esempio IMAP, POP3 e HTTP, oltre a molti altri, ed è in grado di aprire dei socket grezzi di rete e interagirvi con qualsiasi altro protocollo. PHP supporta lo scambio di dati complessi tra virtualmente tutti i linguaggi di programmazione utilizzati sul Web. Riguardo all'interconnessione, PHP offre supporto per instanziare oggetti Java e utilizzarli in modo trasparente come oggetti PHP \cite{PHP}.

\section{API}
Le APIs sono meccanismi che permettono a due componenti software di comunicare tra loro utilizzando un insieme di definizioni e protocolli. Ad esempio, vengono utilizzate dalle app meteo nello smartphone, le quali scaricano i dati meteorologici giornalieri effettuando una richiesta alle API meteo corrispondenti per poter visualizzare gli aggiornamenti.

API significa \textit{Application Programming Interface}. Nel contesto delle APIs, Applicazione si riferisce a ogni software con una funzione distinta, mentre Interfaccia è un contratto di servizio tra due applicazioni, che definisce come queste due possano comunicare tra loro utilizzando richieste e risposte. La loro documentazione API contiene informazioni su come gli sviluppatori debbbamo strutturare tali richieste e risposte.

L'architettura API è generalmente spiegata in termini di client e server. L'applicazione che invia la richiesta è detta client, mentre quella che manda la risposta è il server. Quindi, nell'esempio di prima, il server è il database meteo del provider, mentre l'app mobile costituisce il client.

\subsection{Tipologie di API}
Esistono quattro diversi modi in cui le APIs possono funzionare, in base a quando e perché sono state create.
\begin{itemize}
    \item \textbf{SOAP APIs}\\
    Queste API utilizzano un protocollo chiamato \textit{Simple Object Access Protocol}, dove client e server si scambiano messaggi tramite XML. Questa tipologia di API è meno flessibile, ma è stata più popolare in passato.
    \item \textbf{RPC APIs}\\
    Queste API sono chiamate \textit{Remote Procedure Calls}: il client completa una funzione (o procedura) sul server, quindi il server invia il corrispondente output al client.
    \item \textbf{Websocket APIs}\\
    Sono un'altra tipologia di API utilizzate nel moderno sviluppo web e utilizzano oggetti JSON per il passaggio dei dati. Un'API di tipo WebSocket supporta la comunicazione a due vie tra le app client e il server. Il server, a sua volta, può inviare messaggi di callback ai client connessi, rendendolo così più efficiente delle API di tipo REST.
    \item \textbf{REST APIs}\\
    Sono le API più popolari e flessibili che possiamo trovare, ad oggi, sul web. Il client invia richieste al server sotto forma di dati, quindi il server utilizza questo input fornito dal client per eseguire funzioni interne e ritorna i dati di output, inviandoli indietro al client.
\end{itemize}

Oltre all'architettura, le API sono classificate anche in base al loro scopo di utilizzo:
\begin{itemize}
    \item \textbf{API Private}\\
    Sono interne a un'azienda e vengono utilizzate solamente per connettere sistemi e dati all'interno di essa.
    \item \textbf{API Pubbliche}\\
    Sono aperte al pubblico e possono essere utilizzate da chiunque. Tuttavia, potrebbe esserci qualche tipo di autorizzazione e costo associato al loro utilizzo, anche se non necessariamente.
    \item \textbf{API Partner}\\
    Sono accessibili solo agli sviluppatori esterni autorizzati, per incoraggiare le partnership tra business (B2B).
    \item \textbf{API Composite}\\
    Combinano due o più APIs per risolvere requisiti di sistema o comportamenti complessi.
\end{itemize}

\subsection{RESTful APIs}
REST significa \textit{Representational State Transfer} e definisce un insieme di funzioni come GET, PUT, DELETE, e altre, che i client possono utilizzare per accedere ai dati del server. Sia client che server si scambiano dati mediante HTTP.

La funzionalità principale delle API di rest è la mancanza di stato (\textit{statelessness}), ovvero i server non salvano i dati del client tra le varie richieste. Le richieste che il client effettua al server sono simili agli URL utilizzati per visitare un sito web in un normale browser. La risposta fornita dal server consiste in \textit{plain data}, senza alcun tipo di rendering grafico, a differenza di una pagina web.

Una Web API, detta anche Web Service API, è una \textit{application programming interface} tra un web server e un browser. Tutti i web services sono API, ma non tutte le API sono servizi web. Le API REST sono un tipo speciale di Web API che utilizza lo stile architetturale standard che abbiamo visto poco fa.

I diversi termini riguardanti le APIs, come Java API o service API, esistono perché storicamente le API furono create prima del World-Wide Web. Le API web moderne sono REST APIs e i termini possono ormai essere utilizzati in modo intercambiabile.

\subsection{Utilizzo delle API}
Le integrazioni delle API sono componenti software che aggiornano automaticamente i dati tra client e server. Alcuni esempi di integrazioni API sono quando la galleria del telefono viene sincronizzata automaticamente sul cloud, o la data e l'ora del portatile si sincronizzano automaticamente, aggiornandosi quando viaggiamo in un altro fuso orario. Possono inoltre essere utilizzate dalle aziende per automatizzare molte funzioni di sistema in maniera efficiente.

Le REST APIs offrono quattro principali benefici:
\begin{enumerate}
    \item \textbf{Integrazione}\\
    Le API sono utilizzate per integrare nuove applicazioni con sistemi software esistenti. Questo accelera la velocità di sviluppo, in quanto ciascuna funzionalità non ha bisogno di essere riscritta da zero, ma possiamo utilizzare le API per sfruttare codice già esistente.
    \item \textbf{Innovazione}\\
    L'arrivo di una nuova app è in grado di cambiare intere industrie, quindi le aziende devono essere in grado di rispondere velocemente e supportare il rilascio rapido di servizi innovativi. Questo può avvenire apportando cambiamenti a livello delle API, senza dover riscrivere l'intero codice.
    \item \textbf{Espansione}\\
    Le API presentano un'opportunità unica per le aziende di soddisfare le esigenze dei clienti su diverse piattaforme. Per esempio, le Maps API permettono l'integrazione di informazioni relative alle mappe su siti web, Android, iOS, e altri. Qualunque azienda può offrire un simile accesso ai propri database interni, utilizzando API gratuite o a pagamento.
    \item \textbf{Facilità di manutenzione}\\
    Una API funge da porta di passaggio tra due sistemi. Ciascun sistema è obbligato ad apportare cambiamenti interni così da non influenzare il funzionamento delle API. In questo modo, ogni cambiamento futuro al codice effettuato da una qualunque delle due parti non impatterà l'altra.
\end{enumerate}

Gli API endpoints sono il punto finale del sistema di comunicazione delle API e includono URL di server, servizi, e altre locazioni specifiche digitali da cui vengono inviate e ricevute le informazioni tra i diversi sistemi. Gli API endpoints sono di importanza critica per le aziende per due ragioni principli:
\begin{enumerate}
    \item \textbf{Sicurezza}\\
    Gli API endpoints rendono il sistema vulnerabile agli attacchi, quindi il monitoraggio delle API è cruciale per prevenire un utilizzo improprio.
    \item \textbf{Prestazioni}\\
    Gli API endpoints, specialmente quelli ad alto traffico, possono causare bottlenecks (colli di bottiglia) e influenzare le performance del sistema.
\end{enumerate}

Tutte le API quindi devono essere rese sicure tramite appropriata autenticazione e monitoraggio. I due modi principali per mettere in sicurezza le REST APIs includono:
\begin{enumerate}
    \item \textbf{Token di autenticazione}\\
    Sono utilizzati per autorizzare gli utenti a effettuare le chiamate alle API, controllano che gli utenti siano correttamente identificati e che abbiano effettivi diritti di accesso per quella particolare chiamata API. Per esempio, quando effettuiamo l'accesso al server email, il client email utilizza token di autenticazione per un accesso sicuro.
    \item \textbf{Chiavi API}\\
    Verificano il programma o l'applicazione che effettua la chiamata alle API. Identificano l'applicazione e assicurano che abbia i diritti di accesso richiesti per effettuare quella particolare chiamata API. Le chiavi API non sono sicure come i tokens, ma permettono di monitorare le API per raccogliere dati sul loro utilizzo. Ad esempio, quando visitiamo un sito web e notiamo una lunga stringa di caratteri e numeri nel suo URL, questa stringa è una chiave API che il sito utilizza per effettuare chiamate alle API interne.
\end{enumerate}

\subsection{Documentazione}
Scrivere una documentazione delle API comprensiva è parte del processo di gestione delle API. Tale documentazione può essere generata automaticamente tramite appositi strumenti o scritta manualmente. Come \textit{best practice}, è buona norma scrivere le spiegazioni in linguaggio semplice e facile da leggere, preferibilmente in inglese. I documenti generati con tools possono diventare verbosi e quindi richiedere modifiche manuali. Inoltre, bisogna utilizzare esempi di codice per spiegare le funzionalità e mantenere la documentazione accurata e aggiornata. Lo stile di scrittura va rivolto ai principianti e bisogna coprire tutte le tipologie di problemi che le API possono risolvere per gli utenti \cite{API_AWS}.

\subsection{Evoluzione delle API}
I framework e le librerie possono cambiare le loro APIs. Migrare un'applicazione alle nuove API è tedioso e distrugge il processo di sviluppo: anche se sono stati proposti alcuni strumenti e idee per risolvere l'evoluzione delle API, la maggioranza degli aggiornamenti viene ancora fatta manualmente. Generalmente, i cambiamenti che causano la rottura delle applicazioni esistenti non sono casuali, ma ricadono in particolari categorie: oltre l'80\% di questi è dovuto a \textit{refactorings}, il che suggerisce che per l'aggiornamento delle applicazioni dovrebbero essere utilizzati dei tool di migrazione basati sul refactoring \cite{API_Evolution}.

\section{JSON}
\textit{JavaScript Object Notation}, o JSON, è un formato standard text-based per rappresentare dati strutturati basati su sintassi JavaScript ad oggetti. Viene comunemente utilizzato per trasmettere dati in applicazioni web, ad esempio per inviare dati dal server al client, così da poterlo visualizzare su una pagina web, o viceversa \cite{JSON, JSON_Docs}.

Anche se JSON ricorda molto la sintassi JavaScript per creare gli oggetti letterali, può essere utilizzato indipendentemente da JS e molti ambienti di programmazione offrono la capacità di leggere (parse) e generare JSON.

JSON esiste sotto forma di stringa, utile quando vogliamo trasmettere dati sulla rete. Quando vogliamo accedere ai dati, bisogna convertirlo in un oggetto JavaScript nativo, ma questo non è un problema in quanto JS offre un oggetto JSON globale che dispone di metodi per la conversione tra i due. Convertire una stringa in un oggetto nativo è detto \textit{deserializzazione}, mentre il processo inverso in cui si converte un oggetto nativo in una stringa è detto \textit{serializzazione}. Una stringa JSON può essere salvata in un suo proprio file, che consiste praticamente in un file di testo avente l'estensione \Verb_.json_ e un tipo MIME di \Verb_application/json_.

All'interno di un oggetto JSON, possiamo includere gli stessi tipi di dati basilari che utilizzeremo in un oggetto JavaScript standard, ovvero stringhe, numeri, array, booleani e altri ancora. Questo ci permette di costruire una gerarchia di dati che può essere anche annidata e permette l'accesso nello stesso modo in cui si accede ai campi di un analogo oggetto \cite{JSON}.

\section{SQL}
\textit{Structured Query Language} (SQL) è un linguaggio di programmazione utilizzato per salvare e processare informazioni all'interno di un database relazionale, ovvero in forma tabellare, con righe e colonne che rappresentano diversi attributi e le varie relazioni tra i valori. Possiamo utilizzare le istruzioni SQL per salvare, aggiornare, rimuovere, cercare e raccogliere informazioni dal database, inoltre SQL permette di mantenere e ottimizzare le prestazioni del database \cite{SQL, MySQL_Docs}.

\subsection{MySQL}
MySQL è un \textit{Relational Database Management System} (RDBMS) open-source molto popolare, utilizzato per salvare e gestire dati in maniera affidabile, prestante, scalabile e facile da usare \cite{MySQL,MySQL_Docs}. Da qui in poi ci riferiremo a MySQL come versione del database SQL.

\subsection{Struttura dei dati relazionali}
MySQL è un database relazionale open source che utilizza SQL per creare e gestire i database, salvando i dati in tabelle di righe e colonne organizzate in schemi. Uno schema definisce come i dati sono organizzati e salvati, descrivendo le relazioni esistenti tra le varie tabelle. Con questo formato, possiamo facilmente salvare, raccogliere e analizzare diversi tipi di dati, inclusi semplice testo, numeri, date, orari e, recentemente, JSON e arrays.

Due capacità principali di MySQL sono il suo supporto alle transazioni ACID e la sua abilità di essere scalabile. ACID significa \textit{Atomicity, Consistency, Isolation, and Durability}, ovvero le quattro proprietà che assicurano che le transazioni nel database siano processate in modo affidabile e accurato. Mediante le transazioni ACID, MySQL garantisce che tutte le modifiche ai dati siano effettuate in maniera coerente e affidabile, anche nel caso in cui avvenga un guasto nel sistema. MySQL è in grado di scalare per supportare database di dimensioni molto grandi, gestendo un alto volume di connessioni concorrenti. Le prestazioni, la facilità d'uso e il costo contenuto, insieme alla sua capacità di scalare affidabilmente, hanno reso MySQL il database open source più popolare al mondo.

MySQL è veloce, affidabile, scalabile e facile da utilizzare. Originariamente venne sviluppato per gestire rapidamente grandi database ed è stato utilizzato da molti anni in ambienti di produzione con requisiti molto elevati. MySQL offre un ampio insieme di funzioni di utilità ed è sviluppato costantemente da Oracle, così da restare al passo con le nuove richieste tecnologiche e aziendali. La connettività, velocità, e sicurezza di MySQL lo rendono altamente qualificato per l'accesso ai database su Internet. Alcuni dei principali benefici di MySQL includono:
\begin{itemize}
    \item \textbf{Facilità d'uso}\\
    Gli sviluppatori possono installare MySQL nel tempo di qualche minuto e il database è facile da gestire.
    \item \textbf{Affidabilità}\\
    MySQL è uno dei database più maturi e ampiamente utilizzati, ed è stato testato in una grande varietà di scenari per quasi 30 anni, incluse molte delle aziende più grosse al mondo, le quali dipendono da MySQL per eseguire applicazioni critiche, data la sua affidabilità.
    \item \textbf{Scalabilità}\\
    MySQL è in grado di scalare per soddisfare i bisogni delle applicazioni con il maggior numero di accessi. La sua architettura a replicazione nativa permette alle aziende, tra cui Facebook, Netflix e Uber, di scalare le applicazioni per supportare decine di milioni di utenti, se non di più.
    \item \textbf{Prestazioni}\\
    MySQL è un sistema di database ad alte prestazioni e con zero necessità di amministrazione, viene rilasciato in diverse edizioni per soddisfare quasi tutte le richieste.
    \item \textbf{Elevata disponibilità}\\
    MySQL offre un insieme completo di tecnologie di replicazione native e pienamente integrate per permettere una grande disponibilità e al contempo il recupero dei disastri. Per le applicazioni di business critiche e gli accordi di servizio, i clienti possono raggiungere l'obiettivo di zero perdite di dati, con un tempo di recupero istantaneo.
    \item \textbf{Sicurezza}\\
    La sicurezza dei dati riguarda sia la protezione degli stessi dati, che il rispetto delle leggi dell'industria e del governo, ingluso il GDPR nell'Unione Europea. MySQL Enterprise è in grado di offrire funzionalità di sicurezza avanzate, come l'autenticazione/autorizzazione, la crittografia trasparente dei dati, l'auditing, il data masking e un firewall per il database.
    \item \textbf{Flessibilità}\\
    Il Document Store di MySQL offre agli utenti una massima flessibilità nel sviluppare tradizionali applicazioni database che utilizzino sia SQL che NoSQL, queste ultime senza schemi. Gli sviluppatori possono mischiare tabelle relazionali con documenti JSON nello stesso database e nella stessa applicazione.
\end{itemize}

\subsection{Utilizzo nelle applicazioni web}
MySQL è un database utilizzato largamente nel backend di molte applicazioni web, data la sua abilità di gestire grandi dataset e query complesse in modo veloce e affidabile.

\subsection{Normalizzazione dei dati}
La normalizzazione è il processo di organizzazione dei dati all'interno di un database. Include la creazione di tabelle e lo stabilimento di relazioni tra queste in base a regole definite sia per proteggere i dati, sia per rendere il database più flessibile, eliminando la ridondanza e le dipendenze non coerenti.

I dati ridondanti sprecano spazio su disco e danno luogo a problemi di mantenimento. Se abbiamo bisogno di cambiare dei dati che esistono in più di un posto, tutti i dati vanno modificati nella stessa identica maniera in tutte queste locazioni. Una modifica è invece più facile da implementare se i relativi dati esistono solo in un punto e da nessun'altra parte nel database. Una dipendenza incoerente rende i dati difficili da accedere, in quanto il percorso per trovare tali dati potrebbe essere mancante o malfunzionante.

Esistono alcune regole per la normalizzazione di un database e ciascuna di essere è chiamata una \textit{forma normale}. Generalmente si applicano le prime tre regole per ottenere un database in terza forma normale.

Non sempre però si riesce a rispettare tali regole alla lettera: in generale, la normalizzazione richiede la creazione di tabelle aggiuntive e questo talvolta può risultare scomodo. In questo modo, tuttavia, si generano possibili problemi quali dati ridondanti e dipendenze funzionali \cite{Database_Normalization}.

\section{Apache HTTP Server}
Apache HTTP Server, detto anche semplicemente Apache, è un web server gratuito e open-source che distribuisce contenuti web attraverso la rete Internet, ed è rapidamente diventato il più popolare client HTTP sul web. Il nome trova le sue origini nel rispetto delle tribù dei nativi americani, per la loro resilienza e durabilità \cite{Apache, Apache_HTTP_Server_Docs}.

Apache è solo uno dei componenti richiesti nello stack di un'applicazione web per la distribuzione del contenuto. Uno degli stack più comuni per le webapp è LAMP, ovvero Linux, Apache, MySQL e PHP, o il suo analogo WAMP che utilizza Windows invece che Linux. Il sistema operativo, Linux o Windows, gestisce le operazioni dell'applicazione. Apache è il web server che processa le richieste e serve risorse e contenuti web tramite protocollo HTTP. MySQL è il database che contiene tutte le informazioni in un formato che permette di eseguire queries facilmente. PHP è il linguaggio di programmazione che funziona con Apache per aiutare a creare contenuto web dinamico.

Anche se le statistiche effettive possono variare, gran parte delle webapp vengono eseguite su una qualche forma di stack LAMP in quanto è facile da costruire e utilizzabile in modo gratuito. Generalmente, la maggior parte delle webapp tende ad avere un'architettura simile, anche se il loro scopo può variare considerevolmente, e possono beneficiare dell'utilizzo di Firewalls, Load Balancers, Web Servers, Content Delivery Networks (CDN) e Database Servers.

I Firewall aiutano a proteggere l'applicazione web sia da minacce esterne che da vulnerabilità interne, in base a come sono configurati. I Load Balancers aiutano a distribuire il traffico tra i server web che possono gestire le richieste HTTP(S), e qui è dove Apache entra in gioco, o le richieste verso application servers, ovvero i server che gestiscono le funzionalità e il carico di lavoro di una webapp. I Database Servers, invece, si occupano dello storage di risorse e backups. In base all'infrastruttura, il database e l'applicazione possono coesistere sullo stesso server, anche se tendenzialmente sarebbe raccomandabile mantenerli separati.

La rete Internet consiste di molte tecnologie differenti e non tutte sono le stesse. Anche se Apache costituisce senza dubbio uno dei server web più popolari che esistono sulla rete, ci sono molti altri competitor e il panorama è in costante cambiamento. Negli anni '90 e primi 2000, Apache dominava il mercato, servendo più del 50\% dei siti web attivi su Internet. Un'altra opzione era IIS (Internet Information Services) di Microsoft, ma in confronto era meno popolare. Ad oggi, Apache continua a servire una grande porzione di siti web attivi ma la sua quota di mercato si è ridotta dal 50\% a poco sotto il 40\% nel 2018 e NGINX, pur essendo un nuovo web server, si trova al secondo posto con circa il 35\%, mentre Microsoft IIS si aggira tra l'8 e il 10\%. Ogni anno vengono rilasciate nuove webapp con nuovi stack e server, quindi il panorama continua a cambiare.

Apache è considerato un software open source, il che vuol dire che il codice originale è disponibile liberamente per la visione e la collaborazione. Questo lo ha reso molto popolare tra gli sviluppatori, i quali hanno costruito e configurato dei loro moduli per applicare specifiche funzionalità e migliorare quelle già esistenti. Apache è nato nel 1995 ed è responsabile di aver aiutato ad iniziare la crescita di Internet quando era ancora agli albori.

Uno dei vantaggi di Apache è la sua abilità di gestire grandi quantità di traffico con una configurazione minima. Inoltre, è in grado di scalare con facilità e, grazie alla sua funzionalità modulare, permette di essere configurato in base alle proprie esigenze. Possiamo anche rimuovere moduli indesiderati per rendere Apache più leggero ed efficiente.

Alcuni dei moduli più popolari che possono essere aggiunti includono SSL, il supporto alla programmazione lato server (PHP), e le configurazioni di Load Balancing per gestire grandi quantità di traffico. Apache può essere rilasciato su Linux, macOS e Windows, seppur con percorsi processi di installazione diversi.

Altre funzionalità di Apache Web Server includono la gestione di file statici, il caricamento di moduli dinamici, l'indicizzazione automatica, la configurazione di .htaccess, la compatibilità con IPv6, il supporto a HTTP/2, le connessioni FTP, la compressione e decompressioen mediante Gzip, il restringimento della larghezza di banda, il tracciamento della sessione, il load balancing, la riscrittura di URL e la geolocalizzazione basata su indirizzo IP.

Apache comunica sulla rete da client a server utilizzando il protocollo TCP/IP e può essere utilizzato per una grande varietà di protocolli, di cui il più comune è HTTP/S, ovvero \textit{HyperText Transfer Protocol (Secure)}, uno dei principali protocolli sul web.

Il server Apache è configurato tramite file di configurazione, nei quali si utilizzano i moduli per controllarne il comportamento. Di default, Apache si mette in ascolto sugli indirizzi IP configurati nei file di config che si stanno richiedendo. Qui entra in gioco uno dei punti di forza di Apache: mediante la direttiva Listen, è in grado di accettare e ridirezionare traffico specifico per certe porte e domini in base a specifiche combinazioni di richesta indirizzo-porta. Di default, Listen opera sulla porta 80 (HTTP), ma Apache può essere collegato a diverse porte per diversi domini, permettendo a molti siti web e domini diversi di essere messi in hosting sullo stesso server. 
Una volta che un messaggio raggiunge la sua destinazione, invia un messaggio ACK al mittente, indicano che i dati sono arrivati con successo. Se invece avviene un errore nei dati ricevuti, o dei pacchetti sono stati persi durante il transito, allora viene inviato un messaggio NACK che informa il mittente che è necessario ritrasmettere i dati.

Apache HTTP web server è utilizzato da oltre il 67\% dei web server in tutto il mondo, dato che si tratta di ambienti facili da personalizzare, ma anche veloci, affidabili e altamente sicuri. Questo ha reso i server Apache una scelta comune anche nelle migliori aziende \cite{Apache}.

\section{XAMPP}
XAMPP (\textit{X per Cross-Platform, Apache, MySQL, PHP e Perl}) è uno dei web server multipiattaforma più utilizzati e aiuta gli sviluppatori a creare e testare i loro programmi su un web server locale. Venne sviluppato da Apache Friends e il suo codice, open source, può essere revisionato o modificato dal pubblico. Consiste di un Apache HTTP Server, MariaDB/MySQL e un interprete per i diversi linguaggi di programmazione come PHP e Perl \cite{XAMPP, XAMPP_Docs}.

XAMPP aiuta un host locale o un server a testare il suo sito e client tramite computer e laptop prima di rilasciarlo sul server principale. Questa piattaforma fornisce un ambiente adatto a testare e verificare il corretto funzionamento dei progetti basati su Apache, Perl, database MySQL e PHP tramite il sistema dello stesso host. Tra tutte queste tecnologie, Perl è un linguaggio di programmazione utilizzato per lo sviluppo web, PHP è un linguaggio di scripting per il backend, e MariaDB è il database più utilizzato sviluppato da MySQL.

\subsection{Componenti}
XAMPP è utilizzato per simboleggiare la classificazione di soluzioni per tecnologie differenti. Fornisce una base per il testing di progetti basati su diverse tecnologie attraverso un server personale. XAMPP è un insieme di software che contiene un web server chiamato Apache, un sistema di gestione di database chiamato MariaDB e linguaggi di scripting e programmazione come PHP e Perl. Può funzionare su Windows, Linux e macOS.
\begin{itemize}
    \item \textbf{Cross-Platform}\\
    Diversi sistemi locali hanno diverse configurazioni dovute ai sistemi operativi che vi sono installati. I componenti multipiattaforma sono inclusi per aumentare l'utilità e il pubblico a cui si rivolge questa distribuzione, supportando varie piattaforme come Windows, Linux e macOS.
    \item \textbf{Apache}\\
    Si tratta di un web server HTTP multipiattaforma, utilizzato in tutto il mondo per la distribuzione di contenuto web. L'applicazione server è stata resa gratuita per l'installazione ed è utilizzata dalla community di sviluppatori sotto l'egida di Apache Software Foundation. Il server remoto di Apache consegna i file richiesti, le immagini e altri documenti all'utente.
    \item \textbf{MariaDB}\\
    In origine, il DBMS MySQL era parte di XAMPP, ma ad oggi è stato sostituito da MariaDB. Questo è uno dei DBMS più utilizzati, sviluppato da MySQL, e offre servizi online di data storage, manipolazione dati, recupero dati, riordinamento e cancellazione.
    \item \textbf{PHP}\\
    \`E il linguaggio di scripting per backend principalmente utilizzato per lo sviluppo web. PHP permette agli utenti di creare siti web e applicazioni dinamiche. Può essere installato su ogni piattaforma e supporta una grande varietà di DBMS (\textit{Database Management Systems}). Inoltre, è stato implementato utilizzando il linguaggio C. Il nome PHP è stato derivato da \textit{Personal Home Page tools}, il che spiega la sua semplicità e funzionalità.
    \item \textbf{Perl}\\
    \`E una combinazione di due linguaggi dinamici ad alto livello, ovvero Perl 5 e Perl 6. Perl può essere applicato per trovare soluzioni a problemi basati sull'amministrazione di sistema, sviluppo web e networking. Perl permette ai suoi utenti di programmare applicazioni web dinamiche ed è molto flessibile e robusto.
    \item \textbf{phpMyAdmin}\\
    Questo strumento è utilizzato per gestire MariaDB e si occupa dell'amministrazioe di DBMS.
    \item \textbf{OpenSSL}\\
    \`E l'implementazione open-source del protocollo \textit{Secure Socket Layer / Transport Layer Protocol} (SSL / TLS).
    \item \textbf{XAMPP Control Panel}\\
    Questo pannello aiuta a operare e regolare altri componenti di XAMPP.
    \item \textbf{Webalizer}\\
    Questa soluzione software di Web Analytics è utilizzata per i log utenti e fornisce dettagli sull'utilizzo.
    \item \textbf{Mercury}\\
    \`E un sistema di trasporto di email, ovvero un mail server, che aiuta a gestire le email attraverso il web.
    \item \textbf{Tomcat}\\
    \`E un servlet basato su Java che fornisce funzionalità di Java.
    \item \textbf{Filezilla}\\
    \`E un server avente protocollo FTP (\textit{File Transfer Protocol}), che supporta e facilita le operazioni di trasferimento eseguite sui file \cite{XAMPP}.
\end{itemize}

% \subsection{Configurazione e gestione}
% \subsection{Testing e sicurezza}

\section{Git}
Git è uno strumento di controllo delle versioni del codice (VCS, ovvero \textit{Version Control System}) che è diventato pressoché un must negli ecosistemi di sviluppo software. L'abilità di Git di tracciare meticolosamente i cambiamenti a un progetto lo rende uno strumento essenziale per gli sviluppatori che mirano a gestire i loro progetti in modo efficiente \cite{Git, Git_Docs}.

Il controllo delle versioni (\textit{version control}) ci permette di tracciare i cambiamenti al codice di un software. Quindi, la versione distribuita di un software consiste in un insieme di versioni specifiche di ciascuno dei suoi componenti e file di codice, in quanto ciascuno di essi potrebbe essere stato modificato un numero estremamente variabile di volte.

Tenere traccia dei cambiamenti al codice ci permette di rendere più facile l'identificazione dell'origine di un problema, e allo stesso tempo riduce il rischio di conflitti e sovrascrittura di file. Quindi, Git facilita e semplifica il versioning del software precisamente per questo motivo.

Git offre alcune funzionalità chiave che rendono più facile ottimizzare la gestione del codice e la collaborazione tra teams.

\subsubsection{Visualizzazione della storia del progetto}
La storia dei \textit{commit} è un pilastro chiave per tracciare i progressi del progetto su Git ed è quindi il motivo per cui Git offre agli sviluppatori uno storico dettagliato di tutti i cambiamenti apportati al codice.

Per ciascun nuovo commit vengono tracciati gli specifici cambiamenti apportati ai file del progetto, insieme a un breve messaggio di spiegazione scritto dallo sviluppatore che ha effettuato quel cambiamento. Questi elementi aiutano a migliorare la capacità di comunicazione del team, permettendo una comprensione più rapida delle \textit{insertions} e \textit{deletions} che avvengono nel codice.

In aggiunta al monitoraggio dello sviluppo, questo storico ci permette di tornare indietro se necessario, cancellando parte dei cambiamenti oppure effettuando la \textit{fetch} di solo una parte dei cambiamenti da un \textit{branch} all'altro. Questa funzione svolge un ruolo essenziale nel mantenimento della trasparenza, coerenza e qualità del codice di un progetto su Git, insieme al miglioramento della collaborazione all'interno del team di sviluppo e dell'efficienza operativa per risolvere i problemi.

\subsubsection{Maggiore autonomia per i teams}
Un'altra funzionalità essenziale di Git è lo sviluppo distribuito. Grazie alla sua struttura decentralizzata, Git permette ai team di sviluppatori di lavorare simultaneamente allo stesso progetto, fornendo a ogni membro una propria copia del progetto dove ciascun cambiamento apportato può essere tracciato nel versioning. Questo permette loro di lavorare in autonomia su funzionalità specifiche e di ridurre i rischi di conflitti o sovrascrittura. Questo approcio offre una grande flessibilità per gli sviluppatori che possono quindi esplorare idee diverse e sperimentare nuove funzionalità senza interferire col lavoro svolto dai loro colleghi.

Lo sviluppo distribuito migliora anche la resilienza ai guasti del server: in caso di server failure, ciascuna persona possiede una copia del codice su cui possono continuare a lavorare offline. I cambiamenti possono quindi essere sincronizzati una volta che il server ritorna nuovamente disponibile, riducendo il rischio di distruzione del lavoro per i team di sviluppo e i limiti di aggiornamento per i team operazionali.

\subsubsection{Ottimizzazione dei workflow di sviluppo}
Git è in grado di gestire \textit{branches} e di effettuare il \textit{merging}, permettendo ai team di lavorare in parallelo in un modo collaborativo e organizzato. Ogni nuova aggiunta al codice o bugfix può essere sviluppata e testata indipendentemente, per garantire l'affidabilità. Gli sviluppatori quindi possono semplicemente effettuare il \textit{merge} di questi cambiamenti nel main branch del progetto.

Adottando questo approccio, i team possono tracciare l'evoluzione del codice, collaborare facilmente e in modo efficiente, riducendo i conflitti tra versioni differenti e assicurando un'integrazione continua delle funzionalità sviluppate. I teams possono così sviluppare progetti in modo continuativo e agile, mentre rilasciano regolarmente nuove versioni di codice. Questa pratica facilita molto la gestione dei cambiamenti e allo stesso tempo riduce il rischio di errori.

\subsection{Vantaggi di Git}
Git offre numerosi benefici:
\begin{itemize}
    \item \textbf{Gestione del versioning decentralizzata}\\
    Con Git, ogni sviluppatore possiede una copia completa della storia del progetto, permettendo loro di lavorare indipendentemente.
    \item \textbf{Sicurezza}\\
    A differenza di altri VCS, Git assicura l'integrità di tutti gli elementi all'interno del repository mediante un algoritmo crittografico di hash detto \textit{Secure Hash Algorithm} (SHA-1 e SHA-256). Questo algoritmo protegge il codice e la storia del progetto da ogni modifica benevola o malevola. Inoltre, ogni commit (creazione di una nuova versione) può essere automaticamente firmato (GPG) per assicurare la tracciabilità dei cambiamenti. Questo rende Git uno strumento particolarmente sicuro e affidabile, garantendo l'integrità e l'autenticità del codice e della sua storia.
    \item \textbf{Veloce ed efficiente}\\
    Git massimizza l'efficienza durante lo sviluppo e la sua velocità permette agli sviluppatori di svolgere operazioni complesse, come i commit, branching e merging, in un tempo minimo, anche su grandi codebase. Assicura inoltre un impatto minimo sull'hard disk e anche durante gli scambi su rete. Questa efficienza si traduce in tempi di risposta rapidi durante i backup, le consultazioni e i cambiamenti della storia del progetto.
    \item \textbf{Maggiore flessibilità di lavoro}\\
    Git supporta una grande varietà di workflow di sviluppo, dai modelli centralizzati fino agli approcci lineari. Questa abilità di gestire diversi workflow fornisce ai team numerose opzioni per le possibilità con cui possono lavorare.
    \item \textbf{Facilità di integrazione}\\
    Git eccelle nella sua abilità di integrarsi con una grande varietà di strumenti e piattaforme di sviluppo. Questa larga compatibilità permette ai team di gestire i progetti più efficientemente, sfruttando i migliori tool e pratiche di DevSecOps.
    \item \textbf{Progetto open-source molto popolare}\\
    Git viene supportato da una community dinamica e dedicata, che assicura il suo miglioramento costante. Questa partecipazione attiva da individui e aziende garantisce la regolare aggiunta di nuove funzionalità e miglioramenti attraverso aggiornamenti continui.
\end{itemize}

\subsection{Principali comandi di Git}
Git offre una grande varietà di comandi per rendere più facile il lavoro di squadra, quelli più comuni sono:
\begin{itemize}
    \item \Verb_git init_ inizializza un nuovo repository Git.
    \item \Verb_git clone [url]_ clona un repository esistente.
    \item \Verb_git add[file]_ aggiunge un file all'indice.
    \item \Verb_git commit_ valida i cambiamenti apportati.
    \item \Verb_git commit -m "message"_ valida i cambiamenti con un messaggio.
    \item \Verb_git status_ visualizza lo status dei file nella \textit{working directory}.
    \item \Verb_git push_ invia i cambiamenti al repository remoto.
    \item \Verb_git pull_ effettua la \textit{fetch} dei cambiamenti dal repository remoto e quindi effettua la \textit{merge} con quelli del repository locale \cite{Git}.
\end{itemize}

\subsection{Piattaforme di hosting per repositories}
Un repository Git può essere messo in hosting in diverse piattaforme, tra cui le principali sono \href{https://github.com/}{GitHub}, \href{https://gitlab.com/}{GitLab} e \href{https://bitbucket.org/}{BitBucket}, le quali ospitano repository Git rendendoli accessibili pubblicamente e offrono anche molte delle stesse funzionalità utilizzabili dal terminale con una GUI più user-friendly.
