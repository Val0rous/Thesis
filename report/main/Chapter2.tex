\clearpage{\pagestyle{empty}\cleardoublepage}
\chapter{Tecnologie Utilizzate}
In questo capitolo presenteremo le varie tecnologie utilizzate per la realizzazione del progetto, descrivendole in modo da fornire una conoscenza di base abbastanza solida da permettere la comprensione dell'ambiente.

\section{HTML}
HTML, acronimo che significa \textit{HyperText Markup Language}, è un linguaggio di markup che permette di impaginare e formattare pagine web collegate tra loro tramite link. Un ipertesto non è altro che l'albero di navigazione che collega le pagine web, ovvero un flusso infinito di pagine collegate tra loro attraverso dei link che permettono di spostarsi da un contenuto all'altro.

HTML risponde all'esigenza di riuscire a pubblicare del testo online, mantenendo la formattazione e il significato di ciascuna delle sue parti, utilizzando dei marcatori detti \Verb_<tag>_. Il browser legge il tag e il suo contenuto, quindi traduce a schermo il codice usando i criteri specificati da HTML. Questo ha permesso di costruire pagine con  una struttura simile fra di loro e soprattutto replicabile seguendo uno standard. Nel giro di pochi anni furono aggiunti via via sempre più tag ed elementi per consentire la creazione di pagine contenenti immagini, elementi interattivi, form, pulsanti, tabelle e altri ancora. Pertanto, HTML si è trasformato in un linguaggio molto completo ma in continuo mutamento, che oggi viene mantenuto dal World Wide Web Consortium (W3C), una associazione non governativa che si occupa di implementare nuove funzioni per rendere il web sempre più libero e accessibile.

Riguardo ai tag, sono alla base dell'HTML e ciascuno di essi corrisponde a un determinato tipo di contenuto. Ogni tag può avere degli attributi specifici, cosa che permette di costruire pagine diverse tra di loro e in modo tale che rispondano alle necessità di chi le scrive. Le pagine in HTML hanno una struttura ad albero: proseguendo lungo la ramificazione, si possono trovare più o meno elementi che costruiscono la pagina stessa seguendo una precisa gerarchia. Ad esempio, nel seguente frammento di codice:
\begin{verbatim}
    <html>
        <head>
            <title>Titolo</title>
        </head>
        <body>
            <p>Paragrafo</p>
        </body>
    </html>
\end{verbatim}
\Verb_<html>_ indica l'inizio della parte di codice che verrà espressa utilizzando il linguaggio HTML. Tranne alcuni tag detti \textit{self-closing tags}, tutti vanno chiusi mediante il rispettivo tag di chiusura, che in questo caso è \Verb_</html>_.
\Verb_<head>_ specifica l'header della pagina, il quale racchiude delle informazioni importanti per il suo funzionamento, ma invisibili dal nostro dispositivo. Contiene a sua volta un \Verb_<title>_, ovvero un titolo che è quello rappresentativo della pagina stessa e del suo contenuto. Una volta chiusa l'intestazione, si passa al \Verb_<body>_, cioè il contenuto della pagina. Al suo interno troviamo il tag \Verb_<p>_, utilizzato per scrivere paragrafi di testo, al cui interno viene scritto il rispettivo contenuto. Tutti i vari browser web sono pensati per interpretare i tag più o meno ugualmente ogni volta, ma ci possono essere eccezioni in quanto ogni browser implementa un proprio rendering della pagina web.

HTMl non è un linguaggio di programmazione, bensì un linguaggio di markup: descrive al browser com'è fatta la struttura di una pagina, e niente più. Un linguaggio di programmazine, invece, ha un ruolo funzionale: risolve cicli di codice seguendo una struttura fatta di \Verb_if_ e \Verb_else_, può svolgere calcoli matematici, può manipolare dati e variabili. Quindi, è il browser web che è programmato per capire la struttura delle pagine scritte in HTML, mentre quest'ultimo descrive soltanto la struttura della pagina e del suo contenuto. Di conseguenza, HTML non è un linguaggio di programmazione, mentre lo sono ad esempio JavaScript e PHP.

Ad oggi non è più sufficiente utilizzare solo HTML per realizzare contenuti web, in quanto le esigenze sono da tempo cambiate: ai siti web serve più di soli contenuti testuali, quindi vengono usati anche altri linguaggi, tra cui CSS e JavaScript, che consentono di scrivere pagine ricche e complete, sia nell'aspetto di grafica che nel comportamento \cite{HTML}.

\section{CSS}
CSS, acronimo di \textit{Cascading Style Sheets}, è un linguaggio di stile per i documenti web. I CSS forniscono istruzioni a un browser o a un altro programma utente su come il documento debba essere presentato all'utente, definendone proprietà come il font, i colori, le immagini di sfondo, il layout, il posizionamento delle colonne o di altri elementi della pagina.

CSS è un linguaggio nato per essere il complemento ideale di HTML, per questo i due linguaggi hanno sempre proceduto parallelamente: nelle intenzioni del W3C, HTML è un semplice linguaggio strutturale, estraneo a qualunque scopo che riguardi la presentazione di un documento. Invece, CSS è lo strumento designato proprio per arricchire l'aspetto visuale e la presentazione di una pagina, nato per separare il contenuto dalla presentazione \cite{CSS_Introduzione}.

Anche CSS è un \textit{living standard} come HTML e per questo viene regolarmente sviluppato dal W3C. Il suo compito principale è definire il design di un sito web e per questo scopo vengono assegnati determinati valori agli elementi HTML con l'aiuto delle proprietà del linguaggio di stile. Queste regole hanno una struttura di base che corrisponde allo schema \Verb_selettore { dichiarazione }_. Un selettore non è altro che una rappresentazione dell'elemento HTML al quale fa riferimento la regola, mentre la dichiarazione consiste dalla combinazione tra proprietà e valore che viene annotata tra due parentesi graffe. Ogni dichiarazione finisce con un punto e virgola, come ad esempio \Verb_h2 { color: #ff0000; }_, nel quale il selettore \Verb_h2_ rappresenta le intestazioni di secondo ordine, ovvero i sottotitoli. La dichiarazione, invece, utilizza la proprietà \Verb_color_ per colorare tali sottotitoli di rosso, in questo caso utilizzando un colore espresso in esadecimale per definire impostazioni di colore precise.

CSS supporta diversi selettori in grado di dare istruzioni di regole:
\begin{itemize}
    \item \Verb_selettore_ corrisponde al nome dell'elemento HTML a cui fa riferimento: lo stile viene applicato su tutti gli elementi HTML dello stesso tipo.
    \item \Verb_.selettore_ si rivolge a tutti gli elementi di una classe specifica e si scrive con un punto davanti al nome della classe HTML corrispondente.
    \item \Verb_#selettore_ si riferisce a un unico elemento con un ID specifico. L'integrazione nel codice sorgente HTML avviene grazie all'attributo id, ovvero \Verb_id="identificatore"_.
    \item \Verb_*_ è il selettore universale e si rivolge a tutti gli elementi HTML di un documento.
\end{itemize}

Il CSS può essere dichiarato in vari modi nell'HTML:
\begin{itemize}
    \item Inline: viene definito direttamente sul tag dell'elemento HTML utilizzando l'attributo \Verb_style_. Ha il vantaggio che non è necessario configurare un foglio di stile apposito e ha la priorità elevata, ma nel momento in cui deve essere applicato un blocco di dichiarazioni questo metodo di stile diventa poco chiaro e ridondante.
    \item Interno: viene definito all'interno del tag \Verb_<style>_ nella \Verb_<head>_ del documento HTML.
    \item Esterno: viene definito in un foglio di stile separato, il quale verrà collegato al documento HTML di base utilizzando l'elemento HTML \Verb_<link>_, sempre nella \Verb_<head>_. Può essere riutilizzato in più pagine HTML, semplicemente importandolo ove necessario.
\end{itemize}

\subsection{SCSS}
SCSS, acronimo di \textit{Sassy Cascading Style Sheets}, è un'estensione di CSS molto popolare che aggiunge funzionalità potenti che aumentano le capacità standard di CSS.

In qualità di linguaggio di script preprocessing, SCSS permette agli sviluppatori di utilizzare variabili, regole annidate (\textit{nested}), \textit{mixins} e funzioni, che semplificano il processo di scrittura e mantenimento di stylesheets complessi. Essendo un superset di CSS, SCSS è pienamente compatibile con CSS e rende più facile creare codice pulito, organizzato e riutilizzabile, migliorando l'efficienza e la scalabilità dei progetti di web development.


SCSS è una delle due possibili sintassi per il preprocessore SASS, o \textit{Syntactically Awesome Style Sheets}. Come ogni preprocessore, SASS viene compilato in codice CSS nativo che funziona su ogni web browser. La vera potenzialità di questo linguaggio si vede sul lato sviluppatore, in quanto permette di scrivere codice SCSS conciso che verrà compilato in un codice CSS più lungo. Per questo, gli sviluppatori possono riuscire a fare di più scrivendo meno, senza compromettere la compatibilità col web.

La differenza principale tra SCSS e CSS è che SCSS è una sintassi per il preprocessore SASS, mentre CSS è un linguaggio di stile che descrive come un browser debba visualizzare gli elementi HTML. SCSS supporta da molto tempo l'utilizzo di variabili in qualsiasi punto dello stylesheet, mentre CSS offre un supporto nativo per le variabili che è relativamente recente e può essere usato solo per salvare valori e tokens della UI. In SCSS, la sintassi può essere annidata, permettendo di effettuare il nesting sia delle proprietà che di altri selettori, mentre in CSS un singolo selettore può annidare proprietà ma non altri selettori. Inoltre, SCSS supporta i mixins, che sono gruppi di dichiarazioni CSS che possono essere riutilizzate all'interno dello stylesheet, il che può essere utile quando utilizziamo prefissi vendor (come -webkit- per Chrome o -moz- per Firefox), animazioni complesse, o altro codice che potrebbe essere riutilizzato in più posti.

La sintassi SCSS e CSS hanno delle somiglianze in comune, ma quella SCSS permette l'utilizzo di funzionalità più avanzate. SCSS utilizza i punti e virgola e l'indentazione per mantenere la formattazione, mentre il normale CSS potrebbe essere più flessibile senza tuttavia offrire le funzionalità avanzate di SCSS. L'utilizzo di SCSS permette di mantenere una codebase più pulita e più facilmente mantenibile.

Per compilare SCSS in CSS si utilizza un compiler SASS, che funziona con entrambe le sintassi SASS e SCSS \cite{SCSS}. Queste due sintassi sono equivalenti, l'unica differenza è che in SASS si elimina l'utilizzo di punti e virgola e parentesi graffe mediante una sintassi \textit{whitespace-sensitive}, rendendolo però incompatibile con CSS \cite{SASS}.

\section{JavaScript}
JavaScript è un linguaggio di scripting, o di programmazione, che permette di implementare funzionalità complesse all'interno delle pagine web, ad esempio ogni volta che un sito debba essere dinamico e non statico, per creare contenuti aggiornati automaticamente, mappe interattive, controllare elementi multimediali, oppure grafiche animate in 2D o 3D. Costituisce il terzo strato dello stack delle tecnologie web standard.

Lato client, il linguaggio JavaScript consiste delle solite funzionalità di programmazione come il salvataggio di valori in variabili, le operazioni su stringhe, e l'esecuzione di codice in risposta a certi eventi che avvengono su una pagina web, come quello di \Verb_click_. Ma ancora più importante, ci permette di utilizzare le funzionalità costruite su altro codice JavaScript client-side, mediante le \textit{Application Programming Interfaces}, o APIs. Queste sono blocchi di codice già pronto che consentono a uno sviluppatore di implementare progammi che altrimenti sarebbero difficili o impossibili da implementare, rendendo questi compiti più semplici da realizzare. Le API possono essere built-in nel browser o di terze parti. Queste ultime verranno approfondite in una sezione successiva. Le API del browser permettono di esporre dati dall'ambiente del computer, o effettuare compiti complessi. Per esempio:
\begin{itemize}
    \item DOM (\textit{Document Object Model}) API: permettono di manipolare HTML e CSS, creando, rimuovendo e cambiando l'HTML, applicando nuovi stili dinamicamente alla pagina, ecc. Per esempio, vengono utilizzate ogni volta che vediamo aprirsi una finestra popup su una pagina, o quando viene visualizzato del nuovo contenuto.
    \item Geolocation API: permettono di recuperare informazioni geografiche, come la posizione GPS corrente.
    \item Canvas e WebGL API: permettono di creare animazioni 2D e 3D.
    \item Audio e Video API: permettono di riprodurre audio e video direttamente in una pagina web, o filmare un video dalla webcam e visualizzarlo su un altro computer.
\end{itemize}

JavaScript è un linguaggio leggermente interpretato: il browser riceve il codice JS originale e lancia lo script, e nei browser moderni viene utilizzata una tecnica di compilazione detta \textit{just-in-time} per migliorare le performance. In questo caso, il codice JS viene compilato in un formato binario più veloce mentre lo script è in esecuzione, così da lanciarlo il più rapidamente possibile. Tuttavia, JavaScript è comunque considerato un linguaggio interpretato, dato che la compilazione avviene a \textit{run-time}, invece che \textit{ahead-of-time}.

JavaScript può essere utilizzato sia client-side che server-side. Lato client, viene eseguito sul computer dell'utente, ad esempio quando visualizziamo la pagina web, viene scaricato il codice client-side del sito web, quindi è eseguito e visualizzato dal browser. Lato server, invece, viene eseguito dal server, per poi inviare i risultati al browser lato client, dove verranno scaricati e visualizzati. Tra i vari linguaggi server, possiamo utilizzare JavaScript nell'ambiente Node.js.

Inoltre, è utilizzato per creare pagine web dinamiche, dove \textit{dinamico} si riferisce all'abilità, sia in client- che server-side JS, di aggiornare la visualizzazione di una webpage o webapp per mostrare cose diverse in circostanze diverse, generando il nuovo contenuto richiesto. Il lato server genera dinamicamente il nuovo contenuto, prendendo i dati dal database, mentre lato client JavaScript crea dinamicamente nuovo contenuto all'interno del browser, prendendo i dati che gli sono stati inviati dal backend. Generalmente questi due approcci lavorano insieme. Una pagina web statica, invece, non aggiorna il contenuto dinamicamente e mostra sempre la stessa schermata \cite{JavaScript}.

\subsection{AJAX}
AJAX, acronimo di \textit{Asynchronous JavaScript and XML}, è una tecnica di sviluppo web in cui una webapp scarica contenuto dal server mediante una richiesta HTTP asincrona, quindi lo utilizza per aggiornare le parti della pagina rilevanti, senza bisogno di ricaricare nuovamente la pagina. Questo può rendere la pagina più responsive, in quanto si richiedono solo le parti che necessitano di essere aggiornate.

AJAX può essere utilizzato per creare \textit{single-page apps}, dove l'intera webapp consiste in un singolo documento che usa AJAX per aggiornare il contenuto in base alle richieste.

Inizialmente, AJAX venne implementato utilizzando l'interfaccia \\\Verb_XMLHttpRequest_, ma l'API di \Verb_fetch()_ è più adatta alle applicazioni web moderne, in quanto è più potente, flessibile, e in grado di integrarsi meglio con le tecnologie web fondamentali come i \textit{service workers}. I framework web moderni offrono astrazioni per AJAX. Tuttavia, ad oggi questa tecnica è talmente diffusa che il termine specifico \textit{AJAX} viene utilizzato raramente \cite{AJAX}.

\subsection{Framework e librerie}
\subsection{JSDoc}

\section{Leaflet}

\section{Vue.js}

\section{NPM}

\section{PHP}

\section{API}
\subsection{Tipologie di API}
\subsection{RESTful APIs}
\subsection{Utilizzo delle API}
\subsection{Documentazione}
\subsection{Evoluzione delle API}

\section{JSON}

\section{SQL}
\subsection{MySQL}
\subsection{Struttura dei dati relazionali}
\subsection{Utilizzo nelle applicazioni web}
\subsection{Normalizzazione dei dati}

\section{Apache HTTP Server}

\section{XAMPP}
\subsection{Componenti}
\subsection{Configurazione e gestione}
\subsection{Testing e sicurezza}

\section{Git}
\subsection{Vantaggi di Git}
\subsection{Piattaforme di hosting per repositories}

